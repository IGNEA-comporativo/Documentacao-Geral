%% Generated by Sphinx.
\def\sphinxdocclass{report}
\documentclass[letterpaper,10pt,brazil]{sphinxmanual}
\ifdefined\pdfpxdimen
   \let\sphinxpxdimen\pdfpxdimen\else\newdimen\sphinxpxdimen
\fi \sphinxpxdimen=.75bp\relax
\ifdefined\pdfimageresolution
    \pdfimageresolution= \numexpr \dimexpr1in\relax/\sphinxpxdimen\relax
\fi
%% let collapsible pdf bookmarks panel have high depth per default
\PassOptionsToPackage{bookmarksdepth=5}{hyperref}

\PassOptionsToPackage{booktabs}{sphinx}
\PassOptionsToPackage{colorrows}{sphinx}

\PassOptionsToPackage{warn}{textcomp}
\usepackage[utf8]{inputenc}
\ifdefined\DeclareUnicodeCharacter
% support both utf8 and utf8x syntaxes
  \ifdefined\DeclareUnicodeCharacterAsOptional
    \def\sphinxDUC#1{\DeclareUnicodeCharacter{"#1}}
  \else
    \let\sphinxDUC\DeclareUnicodeCharacter
  \fi
  \sphinxDUC{00A0}{\nobreakspace}
  \sphinxDUC{2500}{\sphinxunichar{2500}}
  \sphinxDUC{2502}{\sphinxunichar{2502}}
  \sphinxDUC{2514}{\sphinxunichar{2514}}
  \sphinxDUC{251C}{\sphinxunichar{251C}}
  \sphinxDUC{2572}{\textbackslash}
\fi
\usepackage{cmap}
\usepackage[T1]{fontenc}
\usepackage{amsmath,amssymb,amstext}
\usepackage{babel}



\usepackage{tgtermes}
\usepackage{tgheros}
\renewcommand{\ttdefault}{txtt}



\usepackage[Sonny]{fncychap}
\ChNameVar{\Large\normalfont\sffamily}
\ChTitleVar{\Large\normalfont\sffamily}
\usepackage{sphinx}

\fvset{fontsize=auto}
\usepackage{geometry}


% Include hyperref last.
\usepackage{hyperref}
% Fix anchor placement for figures with captions.
\usepackage{hypcap}% it must be loaded after hyperref.
% Set up styles of URL: it should be placed after hyperref.
\urlstyle{same}

\addto\captionsbrazil{\renewcommand{\contentsname}{Contents:}}

\usepackage{sphinxmessages}
\setcounter{tocdepth}{2}



\title{Guia de Desenvolvimento}
\date{Aug 29, 2025}
\release{}
\author{P\&D\sphinxhyphen{}IA}
\newcommand{\sphinxlogo}{\vbox{}}
\renewcommand{\releasename}{}
\makeindex
\begin{document}

\ifdefined\shorthandoff
  \ifnum\catcode`\=\string=\active\shorthandoff{=}\fi
  \ifnum\catcode`\"=\active\shorthandoff{"}\fi
\fi

\pagestyle{empty}
\sphinxmaketitle
\pagestyle{plain}
\sphinxtableofcontents
\pagestyle{normal}
\phantomsection\label{\detokenize{index::doc}}


\sphinxstepscope


\chapter{AWS (Lightsail) — Guia de Uso da Plataforma}
\label{\detokenize{sistemas/aws:aws-lightsail-guia-de-uso-da-plataforma}}\label{\detokenize{sistemas/aws::doc}}
\sphinxAtStartPar
Guia para operar a infraestrutura na \sphinxstylestrong{AWS (Lightsail)}.


\bigskip\hrule\bigskip



\section{Conta, Região e Escopo}
\label{\detokenize{sistemas/aws:conta-regiao-e-escopo}}\begin{itemize}
\item {} 
\sphinxAtStartPar
\sphinxstylestrong{Conta \& credenciais}: disponíveis no \sphinxstylestrong{Discord}.

\item {} 
\sphinxAtStartPar
\sphinxstylestrong{Região padrão}: quando possível, \sphinxstylestrong{Brasil (sa\sphinxhyphen{}east\sphinxhyphen{}1)}; caso não, \sphinxstylestrong{N. Virginia (us\sphinxhyphen{}east\sphinxhyphen{}1, Zone A / us\sphinxhyphen{}east\sphinxhyphen{}1a)} por ser o segundo mais próximo.

\item {} 
\sphinxAtStartPar
\sphinxstylestrong{Serviço utilizado}: \sphinxstylestrong{Lightsail} (compute + banco gerenciado).

\end{itemize}


\bigskip\hrule\bigskip



\section{Recursos em uso (hoje)}
\label{\detokenize{sistemas/aws:recursos-em-uso-hoje}}

\begin{savenotes}\sphinxattablestart
\sphinxthistablewithglobalstyle
\centering
\begin{tabulary}{\linewidth}[t]{TTTTT}
\sphinxtoprule
\sphinxstyletheadfamily 
\sphinxAtStartPar
Recurso
&\sphinxstyletheadfamily 
\sphinxAtStartPar
Tipo
&\sphinxstyletheadfamily 
\sphinxAtStartPar
Nome/Identificação
&\sphinxstyletheadfamily 
\sphinxAtStartPar
Região/AZ
&\sphinxstyletheadfamily 
\sphinxAtStartPar
IP
\\
\sphinxmidrule
\sphinxtableatstartofbodyhook
\sphinxAtStartPar
Instância de aplicação
&
\sphinxAtStartPar
Lightsail (Linux)
&
\sphinxAtStartPar
\sphinxstylestrong{GeradorRelatoriosV2}
&
\sphinxAtStartPar
us\sphinxhyphen{}east\sphinxhyphen{}1a
&
\sphinxAtStartPar
\sphinxstylestrong{34.205.142.142} (IP estático)
\\
\sphinxhline
\sphinxAtStartPar
Banco de dados
&
\sphinxAtStartPar
Lightsail Managed DB
&
\sphinxAtStartPar
\sphinxstylestrong{GeradorRelatoriosDatabase}
&
\sphinxAtStartPar
us\sphinxhyphen{}east\sphinxhyphen{}1a
&
\sphinxAtStartPar
Endereço/porta via \sphinxcode{\sphinxupquote{.env}}
\\
\sphinxbottomrule
\end{tabulary}
\sphinxtableafterendhook\par
\sphinxattableend\end{savenotes}
\begin{quote}

\sphinxAtStartPar
\sphinxstylestrong{Aplicação}: 2 contêineres Docker — \sphinxstylestrong{backend + servidor de estáticos} e \sphinxstylestrong{LibreOffice (lo\sphinxhyphen{}runner)}.\\
\sphinxstylestrong{Banco}: \sphinxstylestrong{Lightsail Managed Database}; informações de conexão em \sphinxstylestrong{\sphinxcode{\sphinxupquote{.env}} do Gerador de Relatórios no Discord}.
\end{quote}


\bigskip\hrule\bigskip



\section{Acesso}
\label{\detokenize{sistemas/aws:acesso}}

\subsection{Console AWS}
\label{\detokenize{sistemas/aws:console-aws}}\begin{enumerate}
\sphinxsetlistlabels{\arabic}{enumi}{enumii}{}{.}%
\item {} 
\sphinxAtStartPar
Acesse o \sphinxstylestrong{Console AWS} com as credenciais do \sphinxstylestrong{Discord} como root.

\item {} 
\sphinxAtStartPar
Abra \sphinxstylestrong{Lightsail} \(\rightarrow\) \sphinxstylestrong{Instâncias} \(\rightarrow\) \sphinxstylestrong{GeradorRelatoriosV2}.

\end{enumerate}


\subsection{SSH na Instância}
\label{\detokenize{sistemas/aws:ssh-na-instancia}}\begin{itemize}
\item {} 
\sphinxAtStartPar
\sphinxstylestrong{Usuário}: \sphinxcode{\sphinxupquote{ubuntu}}

\item {} 
\sphinxAtStartPar
\sphinxstylestrong{Autenticação}: chave \sphinxcode{\sphinxupquote{.pem}} (public key) do Lightsail (armazenada na conta/região)
\begin{itemize}
\item {} 
\sphinxAtStartPar
Para mais informações sobre a public key veja {\hyperref[\detokenize{sistemas/aws:publickey}]{\sphinxcrossref{\DUrole{xref}{\DUrole{myst}{Publickey}}}}}.

\end{itemize}

\item {} 
\sphinxAtStartPar
\sphinxstylestrong{Comando (exemplo)}:

\begin{sphinxVerbatim}[commandchars=\\\{\}]
ssh\PYG{+w}{ }\PYGZhy{}i\PYG{+w}{ }/caminho/sua\PYGZhy{}chave\PYGZhy{}lightsail.pem\PYG{+w}{ }ubuntu@34.205.142.142
\end{sphinxVerbatim}

\item {} 
\sphinxAtStartPar
\sphinxstylestrong{Diretório do app}: \sphinxcode{\sphinxupquote{/home/ubuntu/app}}

\end{itemize}
\begin{quote}

\sphinxAtStartPar
Se necessário, use o \sphinxstylestrong{browser\sphinxhyphen{}based SSH} do próprio Lightsail (botão \sphinxstylestrong{Connect} na página da instância).
\end{quote}


\subsection{Uso e Deploy}
\label{\detokenize{sistemas/aws:uso-e-deploy}}
\sphinxAtStartPar
Para informações de uso, deploy e um exemplo de como foi feito esse procesos até agora, veja a documentação do \sphinxhref{https://github.com/IGNEA-comporativo/GeradorDeRelatorios}{Gerador de Relatórios}.


\bigskip\hrule\bigskip



\section{Custos — \sphinxstylestrong{atenção}}
\label{\detokenize{sistemas/aws:custos-atencao}}
\begin{sphinxadmonition}{warning}{Warning:}
\sphinxAtStartPar
\sphinxstylestrong{Monitore custos continuamente.}\\
Instâncias \sphinxstylestrong{Lightsail} e \sphinxstylestrong{bancos gerenciados} cobram \sphinxstylestrong{por hora} caso se ultrapasse os recursos contratados no plano e podem crescer com \sphinxstylestrong{armazenamento}, \sphinxstylestrong{transferência de dados} e \sphinxstylestrong{snapshots}.
\begin{itemize}
\item {} 
\sphinxAtStartPar
\sphinxstylestrong{Acompanhamento}: verificar periodicamente o \sphinxstylestrong{custo mensal} e \sphinxstylestrong{previsão} no Console AWS.

\item {} 
\sphinxAtStartPar
\sphinxstylestrong{Dimensionamento}: valide se o plano da instância e o tamanho do banco estão \sphinxstylestrong{adequados} à demanda.

\item {} 
\sphinxAtStartPar
\sphinxstylestrong{Dados/egresso}: altas transferências podem \sphinxstylestrong{elevar custos}; monitore tráfego e políticas de retenção de logs.

\item {} 
\sphinxAtStartPar
\sphinxstylestrong{Snapshots}: gere com parcimônia e \sphinxstylestrong{apague} os desnecessários.

\end{itemize}
\end{sphinxadmonition}

\sphinxAtStartPar
\sphinxstylestrong{Informações}:
\begin{itemize}
\item {} 
\sphinxAtStartPar
Já existem \sphinxstylestrong{alertas de custo} (Budget) estabelecidos, mas sua manutenção e verificação devem ser mantidos e revisados.

\end{itemize}


\bigskip\hrule\bigskip



\section{Diagrama simples (visão atual)}
\label{\detokenize{sistemas/aws:diagrama-simples-visao-atual}}
\sphinxAtStartPar
Destaca\sphinxhyphen{}se os serviços em uso atual na AWS lightsail.


\bigskip\hrule\bigskip



\section{PublicKey}
\label{\detokenize{sistemas/aws:publickey}}
\sphinxAtStartPar
Para acessar uma instância da AWS Lightsail via SSH ou enviar arquivos via SCP é preciso criar uma \sphinxcode{\sphinxupquote{publickey}}. Essa é a chave que dá acesso a máquina, uma vez que o lightsail não define uma senha pra SSH.

\sphinxAtStartPar
Para criar uma \sphinxcode{\sphinxupquote{publickey}} acesse:
\begin{enumerate}
\sphinxsetlistlabels{\arabic}{enumi}{enumii}{}{.}%
\item {} 
\sphinxAtStartPar
https://lightsail.aws.amazon.com/ls/webapp/account/keys .

\item {} 
\sphinxAtStartPar
Crie uma par de chave de acesso
\sphinxincludegraphics{{publickey_create}.png}

\item {} 
\sphinxAtStartPar
Selecione a região da instância
\sphinxincludegraphics{{publickey_region}.png}

\item {} 
\sphinxAtStartPar
Defina um nome para sua chave e crie
\sphinxincludegraphics{{publickey_name}.png}

\item {} 
\sphinxAtStartPar
Faça um download da chave
\sphinxincludegraphics{{publickey_download}.png}

\item {} 
\sphinxAtStartPar
Você irá baixar um arquivo \sphinxcode{\sphinxupquote{.pem}}.

\end{enumerate}

\sphinxAtStartPar
Você deve deixá\sphinxhyphen{}lo em um lugar de fácil acesso. O caminho padrão (tanto no linux\sphinxhyphen{}wsl quanto no windows) é \sphinxcode{\sphinxupquote{\textasciitilde{}/.ssh/\textless{}nome\_da\_chave\textgreater{}.pem}}, o recomendado é colocá\sphinxhyphen{}la nesse lugar.

\sphinxAtStartPar
Quando for utilizar ssh ou scp faça da seguinte forma:

\begin{sphinxVerbatim}[commandchars=\\\{\}]
ssh\PYG{+w}{ }\PYGZhy{}i\PYG{+w}{ }\PYGZti{}/.ssh/\PYGZlt{}nome\PYGZus{}da\PYGZus{}chave\PYGZgt{}.pem\PYG{+w}{ }\PYGZlt{}ip\PYGZgt{}

scp\PYG{+w}{ }\PYGZhy{}i\PYG{+w}{ }\PYGZti{}/.ssh/\PYGZlt{}nome\PYGZus{}da\PYGZus{}chave\PYGZgt{}.pem\PYG{+w}{ }\PYGZlt{}arquivo1\PYGZgt{}\PYG{+w}{ }\PYGZlt{}arquivo2\PYGZgt{}
\end{sphinxVerbatim}

\sphinxstepscope


\chapter{Contabo — Guia de Uso da Plataforma}
\label{\detokenize{sistemas/contabo:contabo-guia-de-uso-da-plataforma}}\label{\detokenize{sistemas/contabo::doc}}
\sphinxAtStartPar
Guia para operar as VPS da \sphinxstylestrong{Contabo} usadas pelos projetos da equipe. Foca em \sphinxstylestrong{como acessar}, \sphinxstylestrong{onde encontrar} cada projeto e \sphinxstylestrong{o que fazer no dia a dia} pelo \sphinxstylestrong{painel Contabo} e por \sphinxstylestrong{SSH} quando necessário.
\begin{quote}

\sphinxAtStartPar
Para detalhes finos de execução/rotina de cada projeto, consulte a documentação do respectivo sistema.
\end{quote}


\bigskip\hrule\bigskip



\section{O que está em execução}
\label{\detokenize{sistemas/contabo:o-que-esta-em-execucao}}
\sphinxAtStartPar
Temos \sphinxstylestrong{duas VPS separadas}, uma para cada projeto (referenciadas pelo \sphinxstylestrong{nome do projeto}):


\begin{savenotes}\sphinxattablestart
\sphinxthistablewithglobalstyle
\centering
\begin{tabulary}{\linewidth}[t]{TTTTTT}
\sphinxtoprule
\sphinxstyletheadfamily 
\sphinxAtStartPar
Projeto
&\sphinxstyletheadfamily 
\sphinxAtStartPar
VPS/Identificação no painel
&\sphinxstyletheadfamily 
\sphinxAtStartPar
IP
&\sphinxstyletheadfamily 
\sphinxAtStartPar
Diretório principal
&\sphinxstyletheadfamily 
\sphinxAtStartPar
Serviço (\sphinxcode{\sphinxupquote{systemd}})
&\sphinxstyletheadfamily 
\sphinxAtStartPar
Repositório
\\
\sphinxmidrule
\sphinxtableatstartofbodyhook
\sphinxAtStartPar
Monitoramento de Processos
&
\sphinxAtStartPar
vmi2694728
&
\sphinxAtStartPar
213.199.39.53
&
\sphinxAtStartPar
\sphinxcode{\sphinxupquote{/root/ANMScrapping}}
&
\sphinxAtStartPar
\sphinxcode{\sphinxupquote{myscraper.service}}
&
\sphinxAtStartPar
https://github.com/IGNEA\sphinxhyphen{}comporativo/CadastroMineiroScrapping
\\
\sphinxhline
\sphinxAtStartPar
Workana (projeto de terceiro)
&
\sphinxAtStartPar
vmi2713330
&
\sphinxAtStartPar
212.90.121.173
&
\sphinxAtStartPar
\sphinxstyleemphasis{a confirmar}
&
\sphinxAtStartPar
\sphinxstyleemphasis{a confirmar}
&
\sphinxAtStartPar
https://github.com/IGNEA\sphinxhyphen{}comporativo/anm\_api
\\
\sphinxbottomrule
\end{tabulary}
\sphinxtableafterendhook\par
\sphinxattableend\end{savenotes}


\bigskip\hrule\bigskip



\section{Acesso ao painel Contabo}
\label{\detokenize{sistemas/contabo:acesso-ao-painel-contabo}}\begin{itemize}
\item {} 
\sphinxAtStartPar
\sphinxstylestrong{Como acessar}: utilize o \sphinxstylestrong{painel da Contabo} pela \sphinxhref{https://new.contabo.com/servers/vps}{nova versão}, ou pela \sphinxhref{https://my.contabo.com/abos\#}{versão legacy},  com as \sphinxstylestrong{credenciais disponíveis no Discord}.

\item {} 
\sphinxAtStartPar
\sphinxstylestrong{O que você encontra no painel}:
\begin{itemize}
\item {} 
\sphinxAtStartPar
\sphinxstylestrong{Lista de VPS} com status e recursos.

\item {} 
\sphinxAtStartPar
\sphinxstylestrong{Ações rápidas}: reiniciar/desligar/ligar a VPS.

\item {} 
\sphinxAtStartPar
\sphinxstylestrong{Console remoto (NoVNC)} para acesso emergencial.

\item {} 
\sphinxAtStartPar
\sphinxstylestrong{Rede e IP público} (visualização).

\item {} 
\sphinxAtStartPar
\sphinxstylestrong{Histórico de atividades básicas da instância}.

\end{itemize}

\end{itemize}


\bigskip\hrule\bigskip



\section{Acesso por SSH}
\label{\detokenize{sistemas/contabo:acesso-por-ssh}}\begin{itemize}
\item {} 
\sphinxAtStartPar
\sphinxstylestrong{Usuário}: \sphinxcode{\sphinxupquote{root}}

\item {} 
\sphinxAtStartPar
\sphinxstylestrong{Autenticação}: \sphinxstylestrong{senha} disponível no \sphinxstylestrong{Discord}

\item {} 
\sphinxAtStartPar
\sphinxstylestrong{Host}: use o \sphinxstylestrong{IP público} exibido no painel Contabo (e também no discord)

\end{itemize}


\bigskip\hrule\bigskip



\section{Operar os projetos (visão prática)}
\label{\detokenize{sistemas/contabo:operar-os-projetos-visao-pratica}}

\subsection{1) Monitoramento de Processos}
\label{\detokenize{sistemas/contabo:monitoramento-de-processos}}\begin{itemize}
\item {} 
\sphinxAtStartPar
\sphinxstylestrong{Diretório}: \sphinxcode{\sphinxupquote{/root/ANMScrapping}}

\item {} 
\sphinxAtStartPar
\sphinxstylestrong{Serviço}: \sphinxcode{\sphinxupquote{myscraper.service}} (gerencia a execução contínua).

\item {} 
\sphinxAtStartPar
\sphinxstylestrong{Como operar}: utilize os \sphinxstylestrong{alvos do \sphinxcode{\sphinxupquote{Makefile}}} e os passos descritos na \sphinxhref{https://github.com/IGNEA-comporativo/CadastroMineiroScrapping}{documentação do sistema}(reiniciar serviço, verificar status, inspecionar logs, etc.).
\begin{itemize}
\item {} 
\sphinxAtStartPar
Para entendimento do fluxo e rotinas, consulte: \sphinxstylestrong{Documentação do Monitoramento de Processos} (neste repositório) e o repositório do projeto:
\begin{itemize}
\item {} 
\sphinxAtStartPar
GitHub: https://github.com/IGNEA\sphinxhyphen{}comporativo/CadastroMineiroScrapping

\end{itemize}

\end{itemize}

\end{itemize}


\subsection{2) Workana}
\label{\detokenize{sistemas/contabo:workana}}\begin{itemize}
\item {} 
\sphinxAtStartPar
Projeto desenvolvido por \sphinxstylestrong{terceiro contratado via Workana}.

\item {} 
\sphinxAtStartPar
\sphinxstylestrong{Mais informações são necessárias} para operação detalhada (diretório, serviço, comandos).

\item {} 
\sphinxAtStartPar
Repositório: https://github.com/IGNEA\sphinxhyphen{}comporativo/anm\_api

\end{itemize}


\bigskip\hrule\bigskip



\section{Diagrama simples (o que gerenciamos na Contabo)}
\label{\detokenize{sistemas/contabo:diagrama-simples-o-que-gerenciamos-na-contabo}}

\bigskip\hrule\bigskip



\section{Migração obrigatória para AWS (Lightsail/EC2)}
\label{\detokenize{sistemas/contabo:migracao-obrigatoria-para-aws-lightsail-ec2}}
\begin{sphinxadmonition}{warning}{Warning:}
\sphinxAtStartPar
Todos os sistemas atualmente hospedados na \sphinxstylestrong{Contabo} devem ser \sphinxstylestrong{transicionados para a AWS}, preferencialmente como \sphinxstylestrong{contêiner} em \sphinxstylestrong{Lightsail} (quando for “VPS\sphinxhyphen{}like”) ou \sphinxstylestrong{EC2} (quando exigir maior flexibilidade). Essa mudança é \sphinxstylestrong{prioritária} por padronização, segurança, escalabilidade e governança de custos.
\end{sphinxadmonition}

\sphinxAtStartPar
\sphinxstylestrong{Novos serviços}: já nascer na \sphinxstylestrong{AWS} (não criar nada novo na Contabo).

\sphinxstepscope


\chapter{registro.br — Guia de Uso (DNS)}
\label{\detokenize{sistemas/registrobr:registro-br-guia-de-uso-dns}}\label{\detokenize{sistemas/registrobr::doc}}

\section{Acesso ao painel}
\label{\detokenize{sistemas/registrobr:acesso-ao-painel}}\begin{itemize}
\item {} 
\sphinxAtStartPar
Entre no \sphinxstylestrong{painel do registro.br} com as \sphinxstylestrong{credenciais disponíveis no Discord}.

\item {} 
\sphinxAtStartPar
Localize o domínio desejado na sua lista de domínios e clique para abrir os \sphinxstylestrong{detalhes}.

\end{itemize}


\bigskip\hrule\bigskip



\section{Onde adicionar os IPs (configuração DNS)}
\label{\detokenize{sistemas/registrobr:onde-adicionar-os-ips-configuracao-dns}}\begin{enumerate}
\sphinxsetlistlabels{\arabic}{enumi}{enumii}{}{.}%
\item {} 
\sphinxAtStartPar
No domínio escolhido, acesse a área de \sphinxstylestrong{DNS} e clique em \sphinxstylestrong{Editar zona} (ou opção equivalente).

\item {} 
\sphinxAtStartPar
Para apontar o \sphinxstylestrong{endereço IPv4} do seu serviço, \sphinxstylestrong{adicione um registro do tipo “A”}:
\begin{itemize}
\item {} 
\sphinxAtStartPar
\sphinxstylestrong{Nome/Host}: use \sphinxcode{\sphinxupquote{@}} para a raiz do domínio (ex.: \sphinxcode{\sphinxupquote{seu\sphinxhyphen{}dominio.com.br}}) ou informe o subdomínio (ex.: \sphinxcode{\sphinxupquote{www}}).

\item {} 
\sphinxAtStartPar
\sphinxstylestrong{Valor}: informe o \sphinxstylestrong{IPv4} do servidor.

\end{itemize}

\item {} 
\sphinxAtStartPar
\sphinxstylestrong{Salve} as alterações. A atualização pode levar alguns minutos para propagar.

\end{enumerate}

\begin{sphinxadmonition}{note}{Note:}
\sphinxAtStartPar
\sphinxstylestrong{Observação — considerar migração do DNS para a AWS (Lightsail/Route 53)}

\sphinxAtStartPar
Pode ser interessante \sphinxstylestrong{centralizar o DNS na AWS}, especialmente usando o \sphinxstylestrong{Lightsail (DNS integrado)} ou \sphinxstylestrong{Route 53}, para alinhar operação e automação com a infraestrutura já hospedada lá.
\end{sphinxadmonition}



\renewcommand{\indexname}{Index}
\printindex
\end{document}